\documentclass{main}
%----------------------------------------------------------------------------------------
%	Definitions
%----------------------------------------------------------------------------------------
\newcommand{\DefType}{Bachelor Thesis} %Type of your scientific work
\newcommand{\DefTitle}{Modeling and Evaluating CPU Caches in Software} %Title of your work
\newcommand{\DefFirstAuthor}{Mario Theuermann} %Name
\newcommand{\DefFirstAuthorMatnum}{01430751} %Matr number.
\newcommand{\DefFirstAuthorEmail}{mario.theuermann@student.tugraz.at} %Email address
%space for more Authors
%\newcommand{\DefAuthorFront}{\DefFirstAuthor\ \DefFirstAuthorMatnum,\\ \DefSecondAuthor\ \DefSecondAuthorMatnum}
\newcommand{\DefAuthorFrontPage}{\DefFirstAuthor (\DefFirstAuthorMatnum)}

\newcommand{\DefAuthor}{\DefFirstAuthor}
\newcommand{\DefVersion}{1.0}
\newcommand{\DefLogo}{TU_Graz_Logo.pdf}

\begin{document}
\begin{titlepage}
	% --------------------------------------------------------------
	%     Title section & picture
	% --------------------------------------------------------------
	\titlesection{\DefLogo}
	\newcommand{\HRule}{\rule{\linewidth}{0.5mm}} % defines a new command for horizontal lines, change thickness here
	\centering % centre everything on the page
	\vfill \vfill
	
	%------------------------------------------------
	%	Type of scientific document e.g. BS Thesis
	%------------------------------------------------
	\HRule\\[0.4cm]
	{\huge\bfseries\DefType}\\[0.4cm] % Type
	\HRule\\[1.5cm]
	
	%------------------------------------------------
	%	Title of your work e.g. The story of a little fluffball named Heinrich
	%------------------------------------------------
	\textsc{\Large\DefTitle}\\[0.5cm] % Main heading 
	
	\vfill %\vfill
	%------------------------------------------------
	%	Author(s)
	%------------------------------------------------
	%\begin{flushleft}
		\large{
			\DefAuthorFrontPage\\ 	
			\href{mailto:\DefFirstAuthorEmail}{\DefFirstAuthorEmail}\\[0.4cm]
		}
	%\end{flushleft}

	%------------------------------------------------
	%	Date
	%------------------------------------------------
	\vfill\vfill\vfill % Position the date down
	{\large\today} % Date, change the \today to a set date if you want to be precise
	
	%------------------------------------------------
	%	Link
	%------------------------------------------------
	\url{http://www.iaik.tugraz.at/content/teaching/}	
	\vfill % Push the link up 
	
\end{titlepage}

% --------------------------------------------------------------
%     insert last update
% --------------------------------------------------------------
% \lastupdated

% --------------------------------------------------------------
%     abstract
% --------------------------------------------------------------
\begin{abstract}
This file serves as template for the bachelor thesis. \\
\end{abstract}

\pagebreak
% --------------------------------------------------------------
%     TOC
% --------------------------------------------------------------
\tableofcontents

\pagebreak
% --------------------------------------------------------------
%      start document
% --------------------------------------------------------------
\section{Introduction}
As an author, you only have to make minor adjustments to the template:
\begin{itemize}
  \item You may include additional packages in the \textit{main.cls} file. 
  \item You must enter the
  \begin{itemize}
    \item title,
    \item author name(s) and his/her/their matriculation number,
    \item version number, and
    \item topic.
  \end{itemize}
	in \textit{titlepage.tex}.
\end{itemize}

These very few modifications have to be done in the file \textit{titlepage.tex}.
After these modifications, you can start typing the text of your paper in \textit{main.tex}.


\section{The 11 Golden Rules of writing English text}

First of all you are encouraged to deliver your homework in English. In order
to give you some assistance, I provide here some guidelines for writing
English text.

\begin{enumerate}

\item In a sentence, you should try to place noun and verb as close as
possible. In principle, the structure of English sentences is SVO (``Subjekt -
Verb - Objekt'').

\item The beginning of a sentence is typically used to connect it to the
previous text.

\item The end of a sentence is the place to put ``new'' information.

\item The two previous rules also hold for paragraphs, sections, etc.

\item Use passive voice thoughtfully.

\item Don't put too many nouns after each other (remember, English is not
German!).

\item Use so-called ``fill words'' with care. If they are not necessary in the
sense that they make a sentence more readable, then don't use them.

\item Don't make sentences inside sentences. It is better to have several
short sentences after each other, than having one very long sentence.

\item If you use a spell checker, make sure that it is put to only one version
of English.

\item If you define shortcuts for (long) words, make sure that you always use
the same shortcuts throughout your text.

\item Do not use synonyms. They typically do not add information and just
confuse the reader. Furthermore, synonyms are typically rather inexact.

\end{enumerate}

There are several phrases that can be used in the spirit of the second rule.
In the following, I list several useful phrases.

\begin{table}[h]
  \centering
  \begin{tabular}{|l|l|l|}
    % after \\: \hline or \cline{col1-col2} \cline{col3-col4} ...
    \hline
    to give an example & to compare & to summarize\\
    \hline
    for example & also & in other words\\
    for instance & in the same manner & in short\\
    to illustrate the point & similarly & in summary\\
    specifically & likewise & in conclusion\\
    that is & in comparison to & finally\\
    \hline
  \end{tabular}
\end{table}

\begin{table}[htbp]
  \centering
  \begin{tabular}{|l|l|l|}
    % after \\: \hline or \cline{col1-col2} \cline{col3-col4} ...
    \hline
    to emphasize a difference & to emph. a timely relationship & to
    emph. a logical rel. \\
    \hline
    but               & before          & so             \\
    however           & earlier         & if             \\
    yet               & during          & or             \\
    nonetheless       & now             & therefore      \\
    in contrast to    & simultaneously  & consequently   \\
    nevertheless      & meanwhile       & as a result    \\
    still             & at the same time& obviously      \\
    even              & when            & clearly        \\
    though            & while           & if and only if \\
    on the contrary   & later           & it follows that\\
    although          & following       & logically      \\
    despite           & then            & since          \\
    in opposition to  & immediately     & because        \\
                      & thereafter      & nonetheless     \\
                      & after           & by implication   \\
                      & afterwards      & \\
                      & suddenly        & \\
                      & subsequently    & \\
                      & once again      & \\

    \hline
  \end{tabular}
\end{table}

\begin{table}[htbp]
  \centering
  \begin{tabular}{|l|l|l|}
    % after \\: \hline or \cline{col1-col2} \cline{col3-col4} ...
    \hline
    to emphasize a spatial rel. & to emph. a sequence & to emph. a condition \\
    \hline
    below       & and                    & apparently \\
    above       & besides                & seemingly \\
    beneath     & further                & perhaps \\
    beyond      & furthermore            & only when \\
    nearby      & additionally           & if and only if \\
    at hand     & moreover               & it might be possibly \\
    across      & next too first,        & under these circumstances \\
    behind      & second,                & in these cases \\
    throughout  & third,                 & under these conditions \\
    nowhere     & \dots                  & in the case of \\
    everywhere  & firstly,               & \\
                &  secondly,             & \\
                &  thirdly,              & \\
                &  \dots                 & \\
                &  finally               & \\
  \hline
  \end{tabular}
\end{table}

Not all of the phrases, which are listed in the tables, are equally suitable
for a technical English text. For example, it is considered a bad practice to
start a sentence with ``also'' or ``and'' or ``so''. Instead, one of the other
phrase are to be used.

\subsection{Punctuation}

The rules for placing commas and semicolons are, luckily, easier in English
than in German. In order to get most things right, a simple set of rules can
be used as guidance:

\begin{enumerate}

\item No comma before the infinitive: `` This investigation was done \emph{to
get} \dots ''

\item No comma before ``that'': ``Due to the fact that \dots''

\item No comma before indispensable relative clauses: ``The first key was the key which was
suggested most often.''

\item Place a comma before other relative clauses:``The key, which was stored
inside the device, \dots''

\item Place a comma after the phrases listed before: ``Consequently, a comma
must be placed here''. Note that this is not a strict rule. However, either
you always place a comma after such phrases or you never place one.

\item Always place a comma before the last enumeration: ``x, y, and z.''

\end{enumerate}

The semicolon is a stronger separator than the comma, but not as strong as a
period. It should be used thoughtfully. Typically, it substitutes words that
link sentences such as ``and'', ``or'', ``but'', and ``while''.

Of course, there are several books about writing English text. One book that
can be recommended is \cite{DBLP:journals/iacr/SpreitzerP13}.

\section{Frequent Mistakes}

Thanks to my students, there is a (still) growing list of mistakes that I find
frequently in submitted texts. Hence, the text that you submit should not
contain one of the mistakes that are subsequently listed in this section.


\begin{enumerate}

\item joined work should be joint work

\item Possessive in English always has an apostrophe: the device's key. There is only 1 exception: its is without one to avoid confusion with it is (it's cold).

\item ``Proof'' is the noun and ``to prove'' is the verb.

\item Check that all your formulas have ``.'' and ``,'' if needed.

\item Check that captions end with a full stop.

\item Captions should always be above tables and below figures.

\item Make sure that you have some whitespace between the caption and the table

\item Write chapter~V (or Chapter~V), section~IV, table~4, algorithm~1, figure~1, theorem~3 (this will avoid unfortunate linebreaks and pagebreaks in LaTeX)

\item place always a comma after \eg , and \ie\ and if you don't, make sure you write \eg\ and \ie\

\item Latin words should be in \ie , \eg , \etal\

\item In conclusions use past perfect tense.

\item ``practically'' should be ``in practice''

\item ``at a level'' and not ``on a level''

\item ``applied to'' and not ``applied on''

\item ``inexpensive'' sounds much better than ``cheap''

\item avoid ``a lot'' (a substantial amount of, a significant amount of, many)

\item ``use'' should be avoided if possible

\item ``the use of'' sounds better to me than ``the usage of''

\item ``in view of the fact that'' can be replaced by ``since'' or ``as''

\item ``in case of'' should always be ``in the case of''; it is better replaced by ``if'' or ``for''

\item Boolean (not boolean)

\item Hamming (not hamming)

\item 16 bits but 16-bit

\item increase of something: in this case ``increase'' is a noun, therefore, ``of'' must follow

\end{enumerate}

\section{Conclusion}\label{conclusions}
The layout that is used in this document should work without big changes. Note
that this document has been compiled using Miktex (under Windows XP).
Including the hyperref package to get real Links to websites out of the
pdf-version of the paper works more or less as well. Hopefully, the guidelines
for writing in English are useful.


\bibliographystyle{plain}
\bibliography{bibliography}
%\nocite{*}
\end{document}
